\chapter{Overview}
\label{secoverview}
This section presents an overview of the editor. We use an example that
demonstrates how pictures are described and built using {\LEFTY}. This example
is too simple to take advantage of the editor's programmability because the
picture has no structure, but it demonstrates the basic picture-design
principles.

The picture in this example is a collection of rectangles of various sizes
positioned randomly. The complete program for this program can be found in
Appendix~\ref{applistings}.

Picture descriptions consist of two parts:
\begin{list}{}{}
\item[]
data structures that hold information about the picture. For this example, the
data structures tell how many boxes are in the picture and their locations and
sizes.
\item[]
functions that implement operations on the data structures. This example has
functions to insert, delete, move, and draw boxes.
\end{list}

\noindent
Location and size data are stored in {\tt objarray}, which is an array of
key-value pairs, and the number of boxes is given by {\tt objnum}.
Figure~\ref{progboxdsinitial} shows a snapshot of the data structures for two
boxes.  Each element of {\tt objarray} specifies the origin and corner of a box
in fields {\tt rect[0]} and {\tt rect[1]} respectively.
Section~\ref{subseclang} describes the {\LEFTY} language in more detail.
Figure~\ref{figboxinitial} is the WYSIWYG view.

\vspace{-6pt}
\begin{figure}[htb]
\input{progs/progboxdsinitial}
\vspace{-12pt}
\caption{A snapshot of the data structure}
\label{progboxdsinitial}
\end{figure}

\begin{figure}[htb]
\centerline{\hbox{
{\epsfxsize=2.5in \epsffile{figs/figboxinitial.ps}}
}}
\caption{The WYSIWYG view of the picture}
\label{figboxinitial}
\end{figure}

We need functions to draw the picture, to change it, and to bind user events to
changes to the picture. {\tt drawbox} and {\tt redrawboxes} in
Figure~\ref{progboxdraw} do the actual drawing; {\tt drawbox} draws a single
box, and {\tt redrawboxes} clears the display and draws all the boxes. {\tt
box} does the actual rendering; it is a built-in function that draws a
rectangle.  Built-in functions provide access to window and operating system
resources.  They are described in Section~\ref{secbuiltins}. {\tt canvas} is
the id for the drawing area. In this example there is only one drawing area and
its id is assigned to the global variable {\tt canvas}.

\begin{figure}[htb]
\begin{verbatim}
drawbox = function (obj, color) {
    box (canvas, obj, obj.rect, ['color' = color;]);
};
redrawboxes = function () {
    local i;
    clear (canvas);
    for (i = 0; i < objnum; i = i + 1)
        drawbox (objarray[i], 1);
};
\end{verbatim}

\vspace{-12pt}
\caption{Drawing functions}
\label{progboxdraw}
\end{figure}

Figure~\ref{progboxedit} shows various editing functions. {\tt new} adds a new
box to {\tt objarray}. {\tt reshape} changes the shape of an existing box, and
{\tt move} moves a box. {\tt pointadd} is a function that adds two points and
returns the resulting point. {\tt remove} is a built-in function that removes
array elements; here it removes entry {\tt objnum - 1} from {\tt objarray}.

\begin{figure}[htb]
\begin{verbatim}
new = function (rect) {
    objarray[objnum] = [
        'rect' = rect;
        'id' = objnum;
    ];
    objnum = objnum + 1;
    return objarray[objnum - 1];
};
reshape = function (obj, rect) {
    obj.rect = rect;
    return obj;
};
move = function (obj, p) {
    obj.rect[0] = pointadd (obj.rect[0], p);
    obj.rect[1] = pointadd (obj.rect[1], p);
    return obj;
};
delete = function (obj) {
    if (obj.id ~= objnum - 1) {
        objarray[obj.id] = objarray[objnum - 1];
        objarray[obj.id].id = obj.id;
    }
    remove (objnum - 1, objarray);
    objnum = objnum - 1;
};
\end{verbatim}

\vspace{-12pt}
\caption{Editing functions}
\label{progboxedit}
\end{figure}

Figure~\ref{progboxui} shows some of the functions that bind user events to
editing operations. {\tt leftdown} is called when the user presses the left
mouse button. In this picture, if the user presses the left button over white
space (not inside any box) a new---zero size---box is created and the drawing
mode is set to {\tt xor}. As the user moves the mouse while holding the left
button down, {\tt leftmove} is called. {\tt leftmove} creates a rubberband
effect, by reshaping the box so that the corner of the box follows the mouse.
When the user releases the left button, {\tt leftup} is called. {\tt leftup}
sets the drawing mode back to normal (mode {\tt src}). Functions {\tt
middledown}, {\tt middlemove}, and {\tt middleup}, allow the user to grab an
existing box and move it. Finally, {\tt rightup} deletes a box. The names of
these functions are special. When a mouse or keyboard event occurs, {\LEFTY}
searches its data structures for a function that corresponds to the event. If
such a function is defined, {\LEFTY} calls it with one argument, {\tt data}.
{\tt data} is a table that contains information about the user event.  {\tt
data.pos} is the position of the mouse at the time of the event.  {\tt
data.ppos} is present only for move or up events and it holds the position of
the mouse at the time of the corresponding down event. {\tt data.obj} is the
object that the user ``selected''. In this example, each rectangle on the
screen is associated with an entry in {\tt objarray}. This is done by {\tt
drawbox} which calls {\tt box} with the corresponding entry in {\tt objarray}
as the second argument. When the user presses a mouse button over a rectangle,
{\LEFTY} locates that rectangle and from that locates the corresponding object.
{\tt setgattr} is a built-in function that changes the graphics state of a
drawing area. {\tt clearpick} is a built-in function that makes its object
argument unselectable from the WYSIWYG view, by removing it from the data
structure used by {\LEFTY} to locate graphical primitives.

\begin{figure}[htb]
\begin{verbatim}
leftdown = function (data) {
    if (data.obj ~= null)
        return;
    leftbox = new (rectof (data.pos, data.pos));
    drawbox (leftbox, 1);
    setgfxattr (canvas, ['mode' = 'xor';]);
};
leftmove = function (data) {
    if (~leftbox)
        return;
    drawbox (leftbox, 1);
    clearpick (canvas, leftbox);
    reshape (leftbox, rectof (data.ppos, data.pos));
    drawbox (leftbox, 1);
};
leftup = function (data) {
    if (~leftbox)
        return;
    drawbox (leftbox, 1);
    clearpick (canvas, leftbox);
    reshape (leftbox, rectof (data.ppos, data.pos));
    setgfxattr (canvas, ['mode' = 'src';]);
    drawbox (leftbox, 1);
    remove ('leftbox');
};
\end{verbatim}

\vspace{-12pt}
\caption{User Interface functions}
\label{progboxui}
\end{figure}

The user can edit the picture from either the program or the WYSIWYG view. From
the WYSIWYG view, the user can create a new box by pressing the left button,
then---while holding the button down---moving the mouse to another location and
releasing the button. From the program view, the user can perform similar
operations by entering expressions in the editor's language. For example, the
user can create a new box by typing the following commands:

\vbox{
\begin{verbatim}
newbox = new ([0 = ['x' = 250; 'y' = 120;]; 1 = ['x' = 390; 'y' = 240;];]);
drawbox (newbox, 1);
\end{verbatim}
}

The program view shows the data structures and functions of the current
program. By default, entries in this view are shown abstracted, with each entry
taking up a single line. The user can selectively expand individual entries to
their full size.

Figure~\ref{figboxfinal} shows the WYSIWYG view after moving one of the boxes
and Figure~\ref{progboxdsfinal} shows the corresponding data structures.

\begin{figure}[htb]
\input{progs/progboxdsfinal}
\vspace{-9pt}
\caption{A snapshot of the data structure after editing the picture}
\label{progboxdsfinal}
\end{figure}

\begin{figure}[htb]
\centerline{\hbox{
{\epsfxsize=2.5in \epsffile{figs/figboxfinal.ps}}
}}
\caption{The WYSIWYG view of the picture after editing}
\label{figboxfinal}
\end{figure}

Generating postscript output is easy. {\tt canvas} can be set to an id
corresponding to a file. In this case, built-ins such as {\tt box}, will append
the appropriate postscript expressions to this file. Function {\tt dops}, shown
in Figure~\ref{progboxdops} creates a postscript file, sets {\tt canvas} to the
id of the file, draws the picture, closes the file and restores {\tt canvas} to
its original value.

\begin{figure}[htb]
\begin{verbatim}
dops = function () {
    local s;
    s = ['x' = 8 * 300; 'y' = 10.5 * 300;];
    canvas = createwidget (-1, ['type' = 'ps'; 'size' = s;]);
    setwidgetattr (canvas, ['window' = wrect;]);
    redraw (canvas);
    destroywidget (canvas);
    canvas = defcanvas;
};
\end{verbatim}

\vspace{-12pt}
\caption{Function for generating postscript output}
\label{progboxdops}
\end{figure}
