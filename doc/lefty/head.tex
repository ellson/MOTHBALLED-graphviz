\documentstyle[11pt,draft,titlepage]{report}
\input{epsf}
% \epsfverbosetrue

\catcode`@=11
\def\@makechapterhead#1{
    {
        \parindent \z@ \raggedright \reset@font
        \ifnum \c@secnumdepth >\m@ne
            \Large\bfseries \@chapapp{} \thechapter . #1
            \par\nobreak
            \vskip 20\p@
        \fi
        \interlinepenalty\@M
    }
}
\catcode`@=12

\newcommand{\DOT}{{\sl dot}}
\newcommand{\LEFTY}{{\sl lefty}}
\newcommand{\DOTTY}{{\sl dotty}}

\title{Editing Pictures with {\LEFTY}}
\author{Eleftherios Koutsofios}
\newcommand{\lastedited}{96c (09-24-96)}
\date{\lastedited}

\renewcommand{\textfraction}{0.01}
\setlength\oddsidemargin{0.00pt}
\setlength\evensidemargin{0.00pt}
\setlength{\topmargin}{0pt}
\setlength{\headheight}{0pt}
\setlength{\headsep}{0pt}
%\setlength{\footheight}{0pt}
%\setlength{\footskip}{0pt}
\setlength{\textheight}{8.5in}
\setlength{\textwidth}{6.5in}
\begin{document}
\maketitle
\begin{abstract}
{\LEFTY} is a two-view graphics editor for technical pictures. This editor has
no hardwired knowledge about specific picture layouts or editing operations.
Each picture is described by a program that contains functions to draw the
picture and functions to perform editing operations that are appropriate for
the specific picture. Primitive user actions, like mouse and keyboard events,
are also bound to functions in this program. Besides the graphical view of the
picture itself, the editor presents a textual view of the program that
describes the picture. Programmability and the two-view interface allow the
editor to handle a variety of pictures, but are particularly useful for
pictures used in technical contexts, e.g., graphs and trees. Also, {\LEFTY} can
communicate with other processes. This feature allows it to use existing tools
to compute specific picture layouts and allows external processes to use the
editor as a front end to display their data structures graphically. The figure
below shows a typical snapshot of {\LEFTY} in use. The editor has been
programmed to edit delaunay triangulations. The window on the left shows the
actual picture.  The user can use the mouse to insert or move cites and the
triangulation is kept up to date by the editor (which uses an external process
to compute the triangulation).  The window on the right shows the program view
of the picture.

\vspace{0.1in}
\centerline{\hbox{\epsfxsize=5.5in \epsffile{figs/lefty.ps}}}
\end{abstract}

\bibliographystyle{alpha}
