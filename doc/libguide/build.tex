\section{Compiling and linking}
\label{sec:build}
All of the necessary include files and libraries are available
in the {\tt include} and {\tt lib} directories where \gviz\
is installed. At the simplest level, all an application needs
to do to use the layout algorithms is to include {\tt gvc.h},
which provides (indirectly) all of the \gviz\ types and functions,
compile the code,
and link the program with the necessary libraries.

For linking, the application should use the \gviz\ libraries 
\begin{itemize}
\item {\tt gvc}
\item {\tt graph}
\item {\tt cdt}\footnote{
For completeness, we note that it may be necessary to explicitly
link in the following additional libraries, depending
on the options set when \gviz\ was built:
{\tt expat},
{\tt fontconfig},
{\tt freetype2},
{\tt jpeg},
{\tt png},
{\tt z}, and
{\tt ltdl}.
Typically, though, most builds handle these implicitly.}
\end{itemize}

If \gviz\ is built and installed with the GNU build tools, 
there is a {\tt pkg-config} program created in the {\tt bin} 
directory which can be used
to obtain the include file and library information for 
a given installation.
If GNU {\tt make} is used, a sample {\tt Makefile} for building the
programs listed in Appendices~\ref{sec:simple}, \ref{sec:dot} 
and \ref{sec:demo}\footnote{They
can also be found, along with the {\tt Makefile}, in the
{\tt dot.demo} directory of the \gviz\ source.}
could have the form:

\begin{verbatim}
CFLAGS=`pkg-config libgvc --cflags` -Wall -g -O2
LDFLAGS=-Wl,--rpath -Wl,`pkg-config libgvc --variable=libdir` `pkg-config libgvc --libs`

all: simple dot demo

simple: simple.o
dot: dot.o
demo: demo.o

clean:
    rm -rf simple dot demo *.o
\end{verbatim}

