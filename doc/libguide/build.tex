\section{Compiling and linking}
\label{sec:build}
All of the necessary include files and libraries are available
in the {\tt include} and {\tt lib} directories where \gviz\
is installed. At the simplest level, all an application needs
to do to use the layout algorithms is to include {\tt dotneato.h},
which declares all of the \gviz\ types and functions,
compile the code,
and link the program with the necessary libraries.

By convention, for the Windows environment, the libraries and include files
are found in the {\tt lib} and {\tt include} directories,
respectively. On Unix systems, they will occur in  {\tt graphviz/lib} 
and {\tt graphviz/include}. Depending on the what settings are used,
however, these files may occur elsewhere.

For linking, the application should use the \gviz\ libraries 
\begin{itemize}
\item {\tt dotneato}
\item {\tt dotgen}
\item {\tt neatogen}
\item {\tt fdpgen}
\item {\tt twopigen}
\item {\tt circogen}
\item {\tt pack}
\item {\tt common}
\item {\tt gvrender}
\item {\tt pathplan}
\item {\tt gd}\footnote{
As of version 1.14, the {\tt gd} library provided by \gviz\ is essentially
the same as the standard version. Whether or not the \gviz\ version is built
depends on how the software is configured at build time. Also, the \gviz\
version may be named {\tt gvgd}.
}
\item {\tt graph}
\item {\tt cdt}
\end{itemize}
In addition, the following additional libraries should be linked in:
\begin{itemize}
\item {\tt freetype2}
\item {\tt jpeg}
\item {\tt png}
\item {\tt z}
\end{itemize}
For Windows, the \gviz\ binary package provides these latter libraries as
{\tt ft.lib libexpat.lib libexpatw.lib jpeg.lib png.lib z.lib}.
Also, in some environments, it may be necessary to link
in the standard C math library.

If \gviz\ is built and installed with the GNU build tools, 
there is a {\tt dotneato-config} script created in the {\tt bin} 
directory which can be used
to obtain the include file and library information for 
a given installation.
If GNU {\tt make} is used, a sample {\tt Makefile} for building the
programs {\tt dot} and {\tt demo} could have the form:

\begin{verbatim}
COMPILE=libtool --tag=CC --mode=compile ${CC} -c
LINK=libtool --tag=CC --mode=link ${CC}

CFLAGS=`dotneato-config --cflags`
LDFLAGS=`dotneato-config --libtool`

all: simple dot demo

simple: simple.lo
    ${LINK} ${LDFLAGS} -o $@ simple.lo

simple.lo: simple.c
    ${COMPILE} ${CFLAGS} -o $@ simple.c

dot: dot.lo
    ${LINK} ${LDFLAGS} -o $@ dot.lo

dot.lo: dot.c
    ${COMPILE} ${CFLAGS} -o $@ dot.c

demo: demo.lo
    ${LINK} ${LDFLAGS} -o $@ demo.lo

demo.lo: demo.c
    ${COMPILE} ${CFLAGS} -o $@ demo.c

clean:
    rm -rf .libs simple dot demo *.o *.lo
\end{verbatim}

The code for {\tt simple.c}, {\tt dot.c}, and {\tt demo.c} are given in 
Appendices~\ref{sec:simple}, \ref{sec:dot}, and \ref{sec:demo}.\footnote{They
can also be found, along with the {\tt Makefile}, in the
{\tt dot.demo} directory of the \gviz\ source.} They provide three
typical examples of the use of \gviz.

\subsection{Finer-grained usage}

On occasion, the programmer may wish to be more precise as to which \gviz\
include files and libraries are used. For example, the application may need
a function from another library whose name conflicts with a function in an
unused \gviz\ library.
The complete interface is provided by:
\begin{verbatim}
    #include <render.h>
    #include <dotprocs.h>
    #include <neatoprocs.h>
    #include <adjust.h>
    #include <circle.h>
    #include <circo.h>
    #include <fdp.h>
    #include <pack.h>
\end{verbatim}
The file {\tt render.h} provides the basic types and the common
functions, as well as the entry points for format-dependent output.
The files {\tt dotprocs.h}, {\tt neatoprocs.h}, {\tt circle.h},
{\tt circo.h}, and {\tt fdp.h}
define the drawing routines specific to \dot, \neato,
\twopi, \circo, and \fdp, respectively.
The entry points for removing node overlaps is given by {\tt adjust.h}.
Finally, {\tt pack.h} declares the functions for splitting graphs into
connected components, and packing individual layouts together.

Concerning libraries, the {\tt dotgen}, {\tt neatogen}, 
{\tt fdpgen}, {\tt circogen}, and {\tt twopigen} libraries provide the functions
specific to the corresponding layout algorithms.
Note, though, that the \twopi, \circo, and \fdp\ layouts also require the
{\tt neatogen} library.
The {\tt pack} library provides the code for
handling disconnected graphs.
It is not necessary if only the \dot\ algorithm is used.
The {\tt common} library defines
the many shared functions used in all of the algorithms. It requires
the {\tt graph} library, which in turn relies on the {\tt cdt} library.
The {\tt gvrender} library provides an functions for hooking in the
renderers and plug-in libraries.
The {\tt dotneato} library provides the array \verb+char* Info[3];+
which identifies the \gviz\ version and can be used as an argument
when creating a {\tt GVC\_t} value.
Finally, the {\tt pathplan} and {\tt gd} libraries define functions
for edge routing and bitmap rendering. At present, these are usually required
even if the application does not appear to use these features.

Obviously, depending on what is used in the application, various of
the include files and libraries will not be needed.
In particular,
if an application does not plan to use any of the concrete code generators
supplied by \gviz, the software can be built without any of the
optional libraries {\tt jpeg}, {\tt png}, {\tt z},
or {\tt freetype}, and these can be avoided in linking. 
On the other hand, if \gviz\ was built with
some or all of these libraries, linking will have to use
the appropriate libraries.
The {\tt expat} library is used by {\tt common}, though it is possible to
build \gviz\ without this. 

\subsection{Application-specific data}
\label{sec:info}

When possible, the code generators in the library log how an image file
was created in the output file. To do this, they rely on 
the application providing an array 
\begin{verbatim}
    extern char* Info[3];
\end{verbatim}
giving the desired version information.
The three strings should be the name of the application, the version
of the application, and a build date. For example, \dot\ might provide
\begin{verbatim}
    char *Info[] = {
        "dot",              /* Program */
        "1.8.10",           /* Version */
        "6 Dec 2002"        /* Build Date */
    };
\end{verbatim}
A default definition for {\tt Info} is provided in the auxiliary library
{\tt dotneato}, so an
application need only load this library if desired.

